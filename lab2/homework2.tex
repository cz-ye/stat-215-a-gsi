%% Do not edit unless you really know what you are doing.
\documentclass[english]{article}
\usepackage[T1]{fontenc}
\usepackage[latin9]{inputenc}
\usepackage{geometry}
\geometry{verbose,tmargin=1in,bmargin=1in,lmargin=1in,rmargin=1in}
\usepackage{fancyhdr}
\pagestyle{fancy}
\setlength{\parskip}{\smallskipamount}
\setlength{\parindent}{0pt}
\usepackage{url}
\usepackage{enumitem}
\usepackage{amsmath}
\usepackage{amsthm}
\usepackage{amsfonts}
\usepackage{hyperref}
\hypersetup{
    colorlinks=true, 
    linkcolor=blue,
    filecolor=magenta,      
    urlcolor=blue,
}

% define math shortcuts
\DeclareMathOperator{\x}{\mathbf{x}}
\DeclareMathOperator{\y}{\mathbf{y}}
\DeclareMathOperator{\uu}{\mathbf{u}}
\DeclareMathOperator{\vv}{\mathbf{v}}
\DeclareMathOperator{\rr}{\mathbf{r}}
\DeclareMathOperator{\U}{\mathbf{U}}
\DeclareMathOperator{\V}{\mathbf{V}}
\DeclareMathOperator{\A}{\mathbf{A}}
\DeclareMathOperator{\M}{\mathbf{M}}
\DeclareMathOperator{\N}{\mathbf{N}}
\DeclareMathOperator{\D}{\mathbf{D}}
\DeclareMathOperator{\Q}{\mathbf{Q}}
\DeclareMathOperator{\I}{\mathbf{I}}
\DeclareMathOperator{\X}{\mathbf{X}}
\DeclareMathOperator{\W}{\mathbf{W}}
\DeclareMathOperator{\HH}{\mathbf{H}}
\DeclareMathOperator{\R}{\mathbf{R}}
\DeclareMathOperator{\real}{\mathbb{R}}
\DeclareMathOperator{\Xhat}{\vphantom{\mathbf{X}} \smash[t]{\hat{\mathbf{X}}}}
\DeclareMathOperator{\Xbar}{\vphantom{\mathbf{X}} \smash[t]{\bar{\mathbf{X}}}}
\DeclareMathOperator{\bbeta}{\boldsymbol{\beta}}
\DeclareMathOperator{\mmu}{\boldsymbol{\mu}}
\DeclareMathOperator{\llambda}{\boldsymbol{\lambda}}
\DeclareMathOperator{\Lam}{\boldsymbol{\Lambda}}
\DeclareMathOperator{\Delt}{\boldsymbol{\Delta}}
\DeclareMathOperator{\Sig}{\boldsymbol{\Sigma}}
\DeclareMathOperator{\Thet}{\boldsymbol{\Theta}}
\DeclareMathOperator*{\argmin}{\mathrm{argmin} ~ }
\DeclareMathOperator*{\argmax}{\mathrm{argmax} ~ }
\DeclareMathOperator{\Var}{\text{Var}}
\newcommand{\norm}[1]{\lVert #1  \rVert}

\makeatletter
%%%%%%%%%%%%%%%%%%%%%%%%%%%%%% Textclass specific LaTeX commands.
\numberwithin{equation}{section}
\numberwithin{figure}{section}
\newlength{\lyxlabelwidth}      % auxiliary length 

\@ifundefined{date}{}{\date{}}
\makeatother

\usepackage{babel}

\setcounter{section}{-1}



\begin{document}

\title{Homework 2\linebreak{}
Stat 215A, Fall 2023}
\maketitle
\begin{center}
\textbf{Due:} Submit a \texttt{homework2.pdf} file to \textbf{Gradescope} by Friday, October 06 at 11:59pm \\
\par\end{center}

\section{Linear Algebra Review}

Recall that the SVD of $\X \in \real^{n \times p}
$ is a matrix decomposition such that $\X = \U \D \V^{\top}$, where $\U = [\uu_1, \dots, \uu_n] \in \real^{n \times n}$, $\V = [\vv_1, \dots, \vv_p] \in \real^{p \times p}$, and $\D = \text{diag}(d_1, \dots, d_{\min\{ n, p\}}) \in \real^{n \times p}$. In addition, $\U$ and $\V$ are orthogonal matrices so that $\U^{\top} \U = \U \U^{\top} = \I$ and $\V^{\top} \V = \V \V^{\top} = \I$ (i.e., $\uu_j^{\top} \uu_i = \vv_j^{\top} \vv_i = 0$ for all $i \neq j$ and $\uu_j^{\top} \uu_j = \vv_j^{\top} \vv_j = 1$ for all $i$). Moreover, $d_1 \geq \dots \geq d_{\min\{ n, p\}} \geq 0$.

\vspace{2mm}

Now, while the SVD can be used for any rectangular matrix, square matrices have an additional special property and can be decomposed via an eigendecomposition. Given a square matrix $\A \in \real^{p \times p}$, we say that $\vv \in \real^p$ is an \textit{eigenvector} of $\A$ if $\vv \neq \mathbf{0}$ and $\A \vv = \lambda \vv$ for some $\lambda \in \real$. We also call $\lambda$ the \textit{eigenvalue} of $\A$ corresponding to the eigenvector $\vv$. For a more intuitive (geometric) interpretation of eigenvalues and eigenvectors, see this \href{https://www.khanacademy.org/math/linear-algebra/alternate-bases/eigen-everything/v/linear-algebra-introduction-to-eigenvalues-and-eigenvectors}{reference}.

\vspace{2mm}

There is a close connection between the SVD and eigendecomposition. Namely, for any matrix $\X \in \real^{n \times p}$, $\vv \in \real^{p}$ is a right singular vector of $\X$ with singular value $d$ if and only if $\vv \in \real^{p}$ is an eigenvector of $\X^{\top} \X$ corresponding to the eigenvalue $d^2$. You may use this fact without proof.

\section{Principal Components Analysis and SVD}

Let $\X$ be an $n \times p$ data matrix, where $n$ is the number of observations and $p$ is the number of features. For simplicity, we will assume that $\X$ has been mean-centered (i.e., each column of $X$ has mean $0$) and that $n \leq p$.  In the lab section, we used projections in order to introduce the population version of PCA as solving for each $j = 1, \dots, p$
\begin{align} \label{eq:pca_pop}
\vv_j^* = \argmax_{\vv \in \real^p} \vv^{\top} \Var(\X) \vv \qquad \text{subject to} \quad \norm{\vv}_2^2 = 1, \:\:\: \vv^{\top} \vv_i^* = 0 \:\: \forall \: i < j.
\end{align}

However, since $\Var(\X)$ is almost always unknown in practice, we typically estimate $\Var(\X)$ with the sample covariance $\frac{1}{n} \X^{\top} \X$. Thus, in practice, the principal component (PC) directions, $\hat{\vv}_1, \dots, \hat{\vv}_p$, are the solution to the following system of optimization problems:
\begin{align} \label{eq:pca}
\hat{\vv}_j = \argmax_{\vv \in \real^p} \vv^{\top} \X^{\top} \X \vv \qquad \text{subject to} \quad \norm{\vv}_2^2 = 1, \:\:\: \vv^{\top} \hat{\vv}_i = 0 \:\: \: \forall \: i < j.
\end{align}

In this problem, we will take small steps through the proof to show that the PC directions are precisely the right singular vectors of $\X$. 

\begin{enumerate}
\item To begin, prove that the first PC direction $\hat{\vv}_1$ is equal to the first right singular vector $\vv_1$. To show this, use \href{https://en.wikipedia.org/wiki/Lagrange_multiplier}{Lagrange multipliers} to solve the PC1 optimization problem:
\begin{align}
\hat{\vv}_1 = \argmax_{\vv \in \real^p} \vv^{\top} \X^{\top} \X \vv \qquad \text{subject to} \quad \norm{\vv}_2^2 = 1.
\end{align}
If you are not familiar with matrix calculus, \href{https://en.wikipedia.org/wiki/Matrix_calculus}{Wikipedia} is a convenient resource for common derivative identities, which you may find useful here.

\item Next, let $j \in \{2, \dots, p \}$ be given. Use the SVD and matrix multiplication to show that for all $\vv \in \real^p$ satisfying $\vv^{\top} \vv_i = 0$ for each $i < j$, we have
\begin{align}
\vv^{\top} \X^{\top} \X \vv = \sum_{k = j}^{p} \vv^{\top} \left( d_k^2 \vv_k \vv_k^{\top} \right) \vv,
\end{align}
where we define $d_k = 0$ for $k = n+1, \dots, p$.

\item Then, show that for each $j = 2, \dots, p$, the original (sample) PCA formulation in \eqref{eq:pca} is equivalent to 
\begin{align}
\hat{\vv}_j = \argmax_{\vv \in \real^p} \vv^{\top} \left( \X_{(j)}^{\top} \X_{(j)} \right) \vv \qquad \text{subject to} \quad \norm{\vv}_2^2 = 1,
\end{align}
where $\X_{(j)} = \widetilde{\U} \widetilde{\D} \widetilde{\V}^{\top}$, $\widetilde{\U} = [\uu_j, \dots, \uu_n, \uu_1, \dots, \uu_{j-1}] \in \real^{n \times n}$, $\widetilde{\D} = \text{diag}(d_j, \dots, d_n, 0, \dots, 0) \in \real^{n \times p}$, and $\widetilde{\V} = [\vv_j, \dots, \vv_p, \vv_1, \dots, \vv_{j-1}] \in \real^{p \times p}$.

\item Conclude that for each $j = 1, \dots, p$, the $j^{th}$ PC direction, $\hat{\vv}_j$, is equal to the $j^{th}$ right singular vector $\vv_j$. (Hint: Problem 1 may be useful).

\end{enumerate}

\iffalse
\section{Multivariate Statistics}

Suppose that $\x_1, \dots, \x_n$ are n i.i.d. samples from a p-variate normal distribution with known mean $\mathbf{0}$ and unknown covariance $\Sig$, i.e., for each $i = 1, \dots, n$,
\begin{align*}
\x_i \stackrel{i.i.d.}{\sim} \mathcal{N}(\mathbf{0}, \Sig).
\end{align*}

Show that the maximum likelihood estimate of $\Sig$ is given by 
\begin{align*}
\hat{\Sig} = \frac{1}{n} \X^{\top} \X,
\end{align*}
where
\begin{align*}
\X = \begin{pmatrix}
\x_1^{\top} \\
\vdots \\
\x_n^{\top}
\end{pmatrix}
\in \real^{n \times p}
\end{align*}

\textit{Here are some suggested steps to get you started:
\begin{enumerate}
\item Write down the (joint) likelihood for this problem.
\item Calculate the log-likelihood (dropping constant terms as appropriate).
\item Take the gradient of the log-likelihood to get the score equations. Here, it may be easier to parameterize the problem in terms of the precision (inverse covariance) matrix $\Thet = \Sig^{-1}$ and to take the gradient of the log-likelihood with respect to $\Thet$ (instead of $\Sig$).
\item Set the score equations equal to $\mathbf{0}$ and solve for the maximum likelihood estimate $\hat{\Sig}$.
\end{enumerate}
}

\textit{Remark:} Notice that the maximum likelihood estimator $\hat{\Sig}$ under the multivariate normal assumption is precisely the same covariance estimator that we are maximizing in PCA in \eqref{eq:pca}. This is why PCA is optimal and works fantastic with Gaussian data.
\fi




\section{Ordinary Least Squares}

Suppose that we observe our usual data matrix $\X \in \real^{n \times p}$ and response vector $\y \in \real^n$, where $n$ is the number of samples/observations and $p$ is the number of features. Suppose also that $\X$ has rank $p < n$. Under this setting, the ordinary least squares (OLS) estimator is given by

\begin{align*}
\hat{\bbeta}_{OLS} = \argmin_{\bbeta} \norm{\y - \X \bbeta}_2^2.
\end{align*}

\begin{enumerate}
\item Provide an expression for $\hat{\bbeta}_{OLS}$ in terms of $\X$ and $\y$ by solving the optimization problem above. Why do we require the assumption that $\text{rank}(\X) = p < n$?
\item Show that the OLS predictions $\hat{\y} = \X \hat{\bbeta}_{OLS}$ can be written as $\hat{\y} = \HH \y$, where $\HH^2 = \HH$.
%\item How can we interpret $\HH$ using projections?
\item Prove that the residuals $\hat{\rr} = \y - \hat{\y}$ are orthogonal to the OLS predictions $\hat{\y}$. Draw a picture to show what this means geometrically.
\end{enumerate}

\section{Miscellaneous}
\begin{itemize}
    \item What are several concrete actions that could be taken to increase the ability of replication for data based scientific findings?
    \item What are two relevant questions to keep in mind while designing algorithms for partitioning data?
    \item What was the original motivation for the development of the Ridge regression algorithm? What was the original motivation for the development of the LASSO algorithm? 
\end{itemize}


\end{document}

